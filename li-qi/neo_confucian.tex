\documentclass[%
%reprint,
%superscriptaddress,
%groupedaddress,
%unsortedaddress,
%runinaddress,
%frontmatterverbose, 
preprint,
%preprintnumbers,
nofootinbib,
%nobibnotes,
%bibnotes,
amsmath,amssymb,
aps,
longbibliography,
%prd,
%linenumbers,
showkeys,
%prb,
rmp,
%prstab,
%prstper,
%floatfix,
]{revtex4-1}

\usepackage{graphicx}% Include figure files
\usepackage{dcolumn}% Align table columns on decimal point
\usepackage{bm}% bold math
\usepackage{algorithm,algpseudocode}
\usepackage{amsthm}
\usepackage[utf8]{inputenc}
\usepackage{microtype} % Improves typography
\usepackage{hyperref} % Add hypertext capabilities

% Hyperref setup for better links
\hypersetup{
    colorlinks=true,
    linkcolor=blue,
    filecolor=magenta,      
    urlcolor=blue,
    citecolor=blue,
}

\theoremstyle{definition}
\newtheorem{conj}{Conjecture}
\newtheorem{thm}{Theorem}
\newtheorem{lem}{Lemma}
\newtheorem{remark}{Remark}
\newtheorem{defn}{Definition}

\begin{document}

%\preprint{Preprint}

\title{Li and Qi as Theory and Model in Contemporary Mathematics}

\author{Minseong Kim}
\email{mkimacad@gmail.com}

\date{January 5, 2026 (draft)}

\begin{abstract}
In contemporary mathematics---specifically within logic and model theory---a \textit{Theory} is defined as a set of axioms or rules representing essential, core truths. A \textit{Model} is a set of objects that satisfy these axioms. We posit that this relationship maps naturally onto the neo-Confucian concepts of \textit{li} (principle/pattern) and \textit{qi} (material force/object). This paper translates fundamental results from mathematical logic, such as the Lindstr\"om, L\"owenheim-Skolem, and Tennenbaum theorems, into the vocabulary of \textit{li-qi} analysis. By doing so, we demonstrate that the "One Li, Many Qi" debate can be rigorously formalized, revealing that the nature of \textit{qi}'s uniqueness is contingent upon whether the universe is viewed as a computable or non-computable system.
\end{abstract}

\keywords{li-qi, neo-Confucianism, logic, model theory, computability}
\maketitle

\section{Introduction}
The relationship between \textit{li} (principle/pattern) and \textit{qi} (material force/matter) has been the central metaphysical debate in neo-Confucian philosophy for centuries. Historically, this debate has centered on ontological priority and separability: is \textit{li} prior to \textit{qi}? Can \textit{li} exist independently of \textit{qi}, or is \textit{li} merely the immanent order within \textit{qi}? Thinkers such as Zhu Xi argued for a dualistic view where \textit{li} is logically prior to \textit{qi} \cite{zhu1189}, while later scholars like Wang Fuzhi argued for a monistic view where \textit{li} is merely the order of \textit{qi} and cannot exist separately \cite{wang1690}.

In this paper, we propose a novel hermeneutic framework for this ancient debate using contemporary mathematical logic, specifically model theory. In model theory, we distinguish between a \textbf{Theory} (a set of sentences or axioms in a formal language) and a \textbf{Model} (a mathematical structure that satisfies those sentences) \cite{chang90}. We posit that the attributes of a mathematical Theory map remarkably well onto the neo-Confucian concept of \textit{li}, while the attributes of a Model map onto \textit{qi}.

It is important to note at the outset that classical model theory typically analyzes static structures. While neo-Confucian \textit{qi} is often conceptualized as dynamic and generative ("vital force"), this paper focuses on the structural relationship between principle and instantiation. Therefore, we abstract away the temporal dynamics of \textit{qi} to focus on its logical relation to \textit{li}.

By adopting this isomorphism, we can apply rigorous theorems from mathematical logic---such as Lindstr\"om's theorem, the L\"owenheim-Skolem theorems, G\"odel's Incompleteness theorems, and Tennenbaum's theorem---to the philosophical analysis of \textit{li} and \textit{qi}. This approach allows us to formalize questions regarding the uniqueness of \textit{qi} given a specific \textit{li} (the "One Li, Many Qi" problem) and the constraints empirical reality places on abstract principle.

Table \ref{tab:isomorphism}, \ref{tab:philosophy} and \ref{tab:theorems} provide the summaries. For a comprehensive modern review of the metaphysical commitments of Neo-Confucianism, particularly regarding the ontological status of \textit{li} and \textit{qi}, see Angle and Tiwald \cite{angle17} or Liu \cite{liu17b}, as well as \cite{oldstone23}.

\section{Contemporary mathematics and the li-qi analysis}

\subsection{Why contemporary mathematics naturally provides a good analogy for the li-qi analysis}

Model theory is fundamentally the study of the relationship between formal languages (syntax) and their interpretations (semantics) \cite{hodges93}. Syntax consists of symbols and rules of formation without inherent meaning, analogous to pure \textit{li} in its most abstract, pre-material state. Semantics involves mapping these symbols onto a domain of objects---a universe---to assign them truth values, analogous to the instantiation of \textit{li} within concrete \textit{qi}.

The correspondence is not merely superficial. In neo-Confucianism, \textit{li} determines the nature of things, yet it is often described as "empty" or "pure" without the substantiality of \textit{qi}. Similarly, a mathematical theory defines the properties that a structure must have, yet the theory itself is a linguistic abstraction. The structure (the model) provides the "flesh and blood" realization of those properties.
\begin{table}[ht]
\caption{\label{tab:isomorphism}
Correspondence between Neo-Confucian Metaphysics and Mathematical Logic.}
\centering
\begin{tabular}{llll}
\hline\hline
\textbf{Concept} & \textbf{Confucian} & \textbf{Math} & \textbf{Attributes} \\
\hline
Principle &
\textit{Li} &
Theory (Syntax) &
\parbox[t]{0.45\columnwidth}{Abstract, axiomatic, rules, logical priority} \\

Material Force &
\textit{Qi} &
Model (Semantics) &
\parbox[t]{0.45\columnwidth}{Concrete, structural, objects, instantiation} \\

Relation &
\textit{Li--Qi} &
Satisfaction ($\models$) &
\parbox[t]{0.45\columnwidth}{The structure (\textit{qi}) satisfies the axioms (\textit{li})} \\
\hline\hline
\end{tabular}
\end{table}

\subsection{Lindstr\"om theorem: compactness, downward L\"owenheim-Skolem and first-order logic}
\label{subsec:lindstrom}

The most common version of the Lindstr\"om theorem states that first-order logic is the most powerful logic possessing both the compactness property and the downward L\"owenheim-Skolem property (given certain standard assumptions about logic) \cite{lindstrom69}. Understanding these two properties sheds light on the epistemological limits of \textit{li}.

\textbf{Compactness} in logic states that a set of sentences (axioms) has a model if and only if every finite subset of it has a model. In the context of \textit{li-qi}, let us interpret a finite subset of axioms as the limited "local principles" we discern from finite empirical observation. Compactness guarantees that if our finite, local observations are consistent (i.e., they have local \textit{qi} realizations), then they can be extended to a global \textit{li} that governs the entire universe. This provides a logical foundation for the intelligibility of the cosmos: inductive inquiry is not futile because consistent local patterns mathematically guarantee the existence of a coherent universal \textit{li}.

The \textbf{Downward L\"owenheim-Skolem} property states that if a theory (the full set of \textit{li}) has an infinite model, then it has a countable model \cite{skolem20}. Put simply, even if the "true" universe of \textit{qi} contains elements so vast they are mathematically "inaccessible" (uncountable), there exists a simplified realization of \textit{li} where every element can be enumerated using standard natural numbers (0, 1, 2, ...). 
Specifically, if a model $M$ has cardinality $\kappa$, there is a sub-model $M'$ with any infinite cardinality $\kappa' < \kappa$ that satisfies the theory. In neo-Confucian terms, this implies that understanding \textit{li} does not require access to the non-enumerable vastness of the cosmos; \textit{li} can be fully realized in a countable, accessible \textit{qi} domain.

\subsection{Peano arithmetic and the failure of Categoricity}
\label{subsec:peano}

The central question---does one \textit{li} determine a unique \textit{qi}?---corresponds to the mathematical concept of \textbf{Categoricity}. A theory is categorical if all its models are isomorphic (structurally identical). If \textit{li} were categorical, there would effectively be only "One Qi."

However, Peano Arithmetic (PA), the canonical first-order theory of natural numbers, illustrates the failure of categoricity.
PA has the language $\mathcal{L}_{PA} = \{0,1,+,\times, <\}$. A model $M$ consists of a domain $D$ and interpretations of these symbols. 
Compactness implies that if we append axioms stating the existence of a number $c$ greater than every standard natural number ($\exists x, x>n$ for all $n \in \mathbb{N}$), the resulting theory is still consistent. This leads to **non-standard models** containing "infinite" integers. Because PA admits both standard and non-standard models, it is not categorical. 

This proves that if \textit{li} is framed in first-order logic, \textit{li} is structurally incapable of pinning down a unique \textit{qi} universe; "model multiplicity" is unavoidable.

Furthermore, **G\"odel's First Incompleteness Theorem** provides a crucial constraint. G\"odel showed that for any recursive set of axioms capable of arithmetic (our \textit{li}), there exist statements that are true in the standard model (our \textit{qi}) but unprovable from the axioms.
In neo-Confucian terms, this implies a fundamental epistemic gap: even if we possess the correct set of principles (\textit{li}), the actual manifestation of the universe (\textit{qi}) contains truths (semantic facts) that transcend those principles' deductive power (syntactic proofs). Thus, \textit{qi} possesses a semantic richness that \textit{li} cannot exhaust.

Attempts to fix this via induction fail in first-order logic. The induction axiom:
\begin{multline}
\forall P\subseteq \mathbb{N'}: \Big( (0\in P \land \forall n\in \mathbb{N}': (n\in P \rightarrow (n+1)\in P)) \\ 
\rightarrow \forall n\in \mathbb{N}': n\in P \Big)
\end{multline}
requires quantifying over all properties $P$ (subsets). Since first-order logic cannot quantify over subsets, we rely on an axiom schema restricted to definable properties. This leaves uncountably many properties of \textit{qi} undefined by \textit{li}, cementing the disconnect between the single theory and its multiple possible models.

\subsection{Tennenbaum's theorem: Mechanism vs. Vitalism}
\label{subsec:tennenbaum}
There is, however, a significant twist. Tennenbaum's theorem states that the standard model of natural numbers is the \textit{only} countable model of PA in which addition and multiplication are recursive (computable) functions \cite{tennenbaum59}.

This introduces a profound philosophical dichotomy regarding the nature of \textit{qi}:
\begin{enumerate}
    \item \textbf{The Mechanical Universe:} If we postulate that the operations of \textit{qi} follow algorithmic, computable procedures, Tennenbaum's theorem forces uniqueness. The "One Li" of arithmetic pins down exactly "One Qi." This supports a view where \textit{li} and \textit{qi} are inseparable because the principle fully defines the material instantiation.
    \item \textbf{The Vitalist Universe:} If, however, \textit{qi} possesses non-computable qualities—a "vitalism" or spontaneity that transcends algorithmic definition—then the constraints loosen. Non-standard models become possible, re-opening the door to \textit{qi} multiplicity.
\end{enumerate}

Thus, the debate between Monism and Dualism may mathematically hinge on whether one views the action of \textit{qi} as algorithmic (computable) or transcendent (non-computable).

\subsection{How far does first-order logic inform the li-qi analysis}
The analysis of Lindstr\"om and L\"owenheim-Skolem (Section \ref{subsec:lindstrom}) suggests that \textit{li} cannot fully determine \textit{qi}. Essential truths (axioms) are distinguished from accidental truths (model-dependent facts). This distinction supports the orthodox dualism of Zhu Xi: \textit{li} is prior to \textit{qi} because \textit{li} represents the core structure, while \textit{qi} contains contingent variations.

Conversely, if we prioritize empirical reality, we might adopt the set of all true sentences in the physical universe as our theory. Yet, logical limitations suggest that even this "True Arithmetic" has non-standard models. However, if one disregards the logical necessity of alternative models and focuses solely on the empirical instance we inhabit, the \textit{qi}-monist view of Wang Fuzhi becomes consistent: the empirical facts are all that matter.

Or perhaps we should understand \textit{li} and \textit{qi} as providing cross-restrictions on each other—a symbiotic relationship represented by Yulgok Yi I, where \textit{qi} constrains the realization of \textit{li} just as \textit{li} orders \textit{qi} \cite{chung95}. This suggests a dialectic that logic alone may not resolve.

While Tennenbaum's theorem suggests a path to uniqueness via computability, we must remember that Peano Arithmetic is a fragment of mathematics. Zermelo-Fraenkel set theory (ZFC), the foundation of modern mathematics, has no computable models \cite{hamkins13}. If the ultimate \textit{li} of the universe resembles ZFC rather than PA, then computability cannot save us from multiplicity.

\begin{table*}[t]
\caption{\label{tab:theorems}
Mathematical Theorems and their Implications for the \textit{Li--Qi} Relationship.}
\centering
\begin{tabular}{l l l}
\hline\hline
\textbf{Theorem} &
\textbf{Mathematical Content} &
\textbf{Neo-Confucian Interpretation} \\
\hline
Compactness &
\parbox[t]{0.40\textwidth}{Local consistency implies global consistency.} &
\parbox[t]{0.40\textwidth}{Finite local observations of \textit{qi} imply a coherent universal \textit{li}.} \\

L\"owenheim--Skolem &
\parbox[t]{0.40\textwidth}{Infinite models admit countable realizations.} &
\parbox[t]{0.40\textwidth}{\textit{Li} does not require an inaccessible \textit{qi}.} \\

Incompleteness &
\parbox[t]{0.40\textwidth}{True statements exist that are unprovable.} &
\parbox[t]{0.40\textwidth}{Semantic truth (\textit{qi}) exceeds syntactic proof (\textit{li}).} \\

Non-Categoricity &
\parbox[t]{0.40\textwidth}{First-order theories admit non-isomorphic models.} &
\parbox[t]{0.40\textwidth}{One \textit{li}, many possible \textit{qi}.} \\

Tennenbaum &
\parbox[t]{0.40\textwidth}{Computable models of arithmetic are unique.} &
\parbox[t]{0.40\textwidth}{Mechanical \textit{qi} collapses multiplicity into unity.} \\
\hline\hline
\end{tabular}
\end{table*}

\begin{table}[h]
\caption{\label{tab:philosophy}
Historical Positions Reframed via Logic.}
\centering
\begin{tabular}{l l}
\hline\hline
\textbf{Stance} & \textbf{Logical Counterpart} \\
\hline
Zhu Xi (Dualism) &
\parbox[t]{0.65\columnwidth}{Theory precedes model; axioms underdetermine structure.} \\

Wang Fuzhi (Monism) &
\parbox[t]{0.65\columnwidth}{Only the physical model exists; theory records truths.} \\

Yi Yulgok (Symbiosis) &
\parbox[t]{0.65\columnwidth}{Mutual constraint between axioms and structures.} \\

Computationalism &
\parbox[t]{0.65\columnwidth}{Computability enforces uniqueness (example: Tennenbaum).} \\
\hline\hline
\end{tabular}
\end{table}


\section{Conclusion}

This paper has attempted to reframe the classic neo-Confucian debate regarding \textit{li} and \textit{qi} through the lens of modern model theory. We have shown that if we accept the mapping of \textit{li} to formal Theory and \textit{qi} to Model, the mathematical properties of logic provide a rich vocabulary for articulation. 

The results of Lindstr\"om and L\"owenheim-Skolem suggest that a pure principle (\textit{li}) is fundamentally incapable of uniquely determining a material reality (\textit{qi}), lending credence to the dualist separation. G\"odel's Incompleteness Theorem further highlights that the semantic truth of \textit{qi} always overflows the syntactic proofs of \textit{li}. However, Tennenbaum's theorem suggests that if reality is fundamentally computational, the "One Li, Many Qi" problem collapses into uniqueness.

Finally, we must acknowledge that classical model theory treats structures as static. Neo-Confucian \textit{qi}, however, is inherently dynamic and generative. While this paper has successfully analyzed the structural relationship between principle and instantiation, future research should extend this analysis using dynamic logic or category theory to better capture the transformative nature of \textit{qi}, moving beyond the static \textit{li-qi} isomorphism presented here.

\section*{Data availability and declaration of interests}

The author(s) have no funding source to declare. Furthermore, there is no conflict of interests. 

\bibliography{infoamp2}% Produces the bibliography via BibTeX.

\end{document}
